

 Contactamos con ustedes, con el objetivo de exponer la solicitud de patrocinio para la \textit{XVI Semana Linux} de la Universidad Distrital Francisco Jos\'e de Caldas que daremos inicio el pr\'oximo 26 de Agosto del presente año con el tema de  \textit{Inteligencia Artificial}.  \\ 


La Semana Linux es un evento anual de visibilidad nacional, organizado por el \textit{Grupo GNU Linux de la Universidad Distrital} (GLUD), en el cual se da a conocer el Software Libre y sus ventajas para el progreso tecnol\'ogico de la sociedad, all\'i convergen estudiantes, profesores, expertos y usuarios finales interesados en familiarizarse con el uso del Software Libre y su filosofía.


Durante la semana que comprende desde el 26 de agosto hasta el 31 de Agosto se presentan conferencias, talleres y actividades acerca de diferentes temas enfocados a promover la tecnología, software, cultura y ciencia libre, contando as\'i con el sector privado y p\'ublico, ademas de exponentes que hacen parte de universidades y comunidades de la ciudad.  De esta forma, nos complace invitar a \textit{Correlibre} como patrocinador del evento haci\'endolo participe en las actividades de nuestro evento. 

Con su patrocinio ofrecemos:

\begin{itemize}
    \item Espacio para conferencias que deseen presentar como compañía, todo enfocado en uso de herramientas y Software Libre.
    \item Exposición de su marca, logo y publicidad en redes sociales del grupo GNU/LInux UD  GLUD y en el sitio web del evento \textit{semana.glud.org} 
    \item Registro de asistentes al evento para b\'usqueda de talentos que puedan servir en las b\'usquedas de staff de su compañía.
\end{itemize}


Teniendo en cuenta que la Semana Linux es un evento organizado por una comunidad acad\'emica sin animo de lucro, y los costos del evento son de diferentes tipos, creemos necesario exponer nuestras necesidades, no sin antes resaltar que el patrocinio puede ser económico y/o dando respuesta a alguna necesidad que expondremos a continuación: 


Para el desarrollo de las conferencias y la logística de la  semana requerimos:

\begin{table}
\begin{center}
\begin{tabular}{|l|l|}
\hline
Descripción & Costo \\
\hline \hline
Impresión de productos publicitarios como volantes, \\  afiches, stickers y manillas  & \$200.000 \\ \hline
Bebida y refrigerios para los conferencistas durante toda la semana & \$150.000 \\ \hline
Camisetas y escarapelas para el personal de log\'istica &  \$300.000 \\ \hline
\end{tabular}
%\caption{}%Tabla muy sencilla.}
%\label{tabla:sencilla}
\end{center}
\end{table}


Con respecto a la actividad de cierre que consiste en un evento presencial de desarrollo colaborativo de software o hardware, Hackathon SLUD 2019, con una duraci\'on de 24 horas, desde el viernes 30 de Agosto hasta el sabado 31 de agosto, requerimos: 

\begin{table}
\begin{center}
\begin{tabular}{|l|l|}
\hline
Descripción & Costo \\
\hline \hline
Cena para los participantes y personal de logistica  viernes \\ 30 de Agosto (Aprox. 30 personas) & \$250.000 \\ \hline
Desayuno para los participantes y personal de logistica  sabado \\ 31 de Agosto (Aprox. 30 personas) & \$150.000 \\ \hline
Almuerzo para los participantes y personal de logistica  viernes \\ 30 de Agosto (Aprox. 30 personas) & \$250.000 \\ \hline
Punto de hidrataci\'on, caf\'e, bebidas y refrigerios \\ para los participantes durante la jornada & \$150.000 \\ \hline

\end{tabular}
%\caption{}%Tabla muy sencilla.}
%\label{tabla:sencilla}
\end{center}
\end{table}

Para esta actividad, todos los años entregamos una premiaci\'on para primer, segundo y tercer puesto de \$500.000, \$200.000 y \$100.000 pesos, respectivamente. Sin embargo los premios del segundo y tercer puesto pueden variar teniendo en cuenta los recursos disponibles.


\begin{table}
\begin{center}
\begin{tabular}{|l|l|}
\hline
Descripción & Costo \\
\hline \hline
Premiaci\'on: Primer Puesto Hackathon & \$500.000 \\ \hline
Premiaci\'on: Segundo Puesto Hackathon & \$200.000 \\ \hline
Premiaci\'on: Segundo Puesto Hackathon & \$100.000 \\ \hline
\end{tabular}
%\caption{}%Tabla muy sencilla.}
%\label{tabla:sencilla}
\end{center}
\end{table}


En caso de patrocinar la premiaci\'on de la Hackathon, a cambio pueden definir el enfoque de la jornada para que atienda alguna necesidad de su empresa siempre y cuando el producto sea software o hardware libre y este en el marco de la semana con el tema \textit{Inteligencia Artificial} 


Para terminar, solicitamos inicialmente el apoyo económico de  quinientos mil pesos colombianos \textbf{\$ 500.000}, para las necesidades con respecto al desarrollo de las conferencias y su logística, sin embargo esta cifra esta sujeta a cambios según los recursos que ustedes nos puedan brindar, de antemano muchas gracias y  esperamos poder contar con su apoyo y asistencia al evento para tener la oportunidad de compartir con ustedes conferencias, conocimiento y oportunidades que permitan dar crecimiento a nuestra comunidad.